\documentclass[a4paper,capchap,espacoduplo,normaltoc]{abntepusp-en} % Change "abntepusp-en" to "abntepusp" for Portuguese.

%\usepackage[bookmarks,pdftex,a4paper,colorlinks=true,citecolor=black,urlcolor=blue,linkcolor=black,pdfpagemode=None]{hyperref}
\usepackage[bookmarks,colorlinks=true,citecolor=black,urlcolor=blue,linkcolor=black,pdfpagemode=UseNone]{hyperref}
\usepackage[centertags]{amsmath}
\usepackage{amsfonts}
\usepackage{amssymb}
\usepackage{amsthm}
\usepackage[T1]{fontenc}
\usepackage[latin1]{inputenc}
\usepackage[english]{babel}  % Change "english" to "brazil" for Portuguese.
\usepackage[alf,abnt-repeated-author-omit=yes]{abntcite}
\usepackage{url}
%\usepackage{winfonts}
\usepackage{txfonts}
\usepackage[ddmmyyyy]{datetime}

%\fontfamily{arial}\selectfont
%\renewcommand{\rmdefault}{arial}

\newcommand{\TODO}[1]{~~\textcolor{red} {\textbf{TODO: #1}}}

% Math -------------------------------------------------------------------
\newtheorem{theorem}{Theorem}{\bfseries}{\itshape}
\newtheorem{lemma}{Lemma}{\bfseries}{\itshape}
\newtheorem{definition}{Definition}{\bfseries}{\itshape}
\newtheorem{corollary}{Corollary}{\bfseries}{\itshape}
\newtheoremstyle{example}{\topsep}{\topsep}%
	{}%         Body font
	{}%         Indent amount (empty = no indent, \parindent = para indent)
	{\bfseries}% Thm head font
	{:}%        Punctuation after thm head
	{.5em}%     Space after thm head (\newline = linebreak)
	{\thmname{#1}\thmnumber{ #2}\thmnote{ #3}}%         Thm head spec
\theoremstyle{example}
\newtheorem{example}{Example}

\sloppy

\begin{document}

% document with one author
\autorPoliI{Name}{Middle}{Lastname}

% document with two authors
%\autorPoliII{Name1}{Middle1}{Lastname1}{Name2}{Middle2}{Lastname2}

% document with three authors
%\autorPoliIII{Name1}{Middle1}{Lastname1}{Name2}{Middle2}{Lastname2}{Name3}{Middle3}{Lastname3}

\titulo{<Title>}

\orientador{<Adviser's Name>}

\relatFapesp
%\monografiaFormatura
%\monografiaMBA
%\qualificacaoMSc{<�rea do Mestrado>}
%\qualificacaoMSc{Enge\-nharia El�trica}
%\dissertacao{<�rea do Mestrado>}
%\qualificacaoDr{<�rea do Mestrado>}
%\teseDr{<�rea do Doutorado>}
%\teseLD
%\memorialLD

%\areaConcentracao{<�rea de Concentra��o>}
\areaConcentracao{Engenharia de Computa��o}

%\departamento{<Departamento>}
\departamento{Departamento de Engenharia de Computa��o e Sistemas Digitais (PCS)}

\local{<Cidade>}

\data{<Ano>}

\dedicatoria{}

\capa{}

\folhaderosto{}

% Ficha Catalogr�fica

%\setboolean{PoliRevisao}{true} % gera o quadro de revis�o ap�s a defesa
\renewcommand{\PoliFichaCatalograficaData}{%
  1. Assunto \#1. 2. Assunto \#2. 3. Assunto \#3.
  I. Universidade de S�o Paulo. Escola Polit�cnica.
  \PoliDepartamentoData. II. t.}

\fichacatalografica % formata a ficha

\paginadedicatoria{}

\begin{agradecimentos}
\end{agradecimentos}

\begin{resumo}
\end{resumo}

\begin{abstract}
\end{abstract}

\begin{resume}
\end{resume}

\begin{zusammenfassung}
\end{zusammenfassung}

\tableofcontents

\listoffigures

\listoftables

\begin{listofabbrv}{1000}
\item [USP] University of S\~ao Paulo
\item [CFS] Courtois-Finiasz-Sendrier
\end{listofabbrv}

\begin{listofsymbols}{1000}
\item [$\Delta(h)$] Dyadic signature
\end{listofsymbols}

\chapter{Introduction}\label{chp:intro}

\section{Presentation}\label{sec:presentation}

\section{Goals}\label{sec:goals}

\section{Original contributions}\label{sec:contrib}

\begin{example}
This is an example.
\qed
\end{example}

\section{Organization}

\chapter{Another chapter}\label{chp:outrocap}

This chapter develops the theory of~\cite{agashe-lauter-venkatesan,al-riyami-paterson,ansi-x9.62,balasubramanian-koblitz,blake-seroussi-smart,bleichenbacher,loftus-may-smart-vercauteren,weimerskirch}.

\[
E = mc^2.
\]

\chapter{Conclusions}\label{chp:conclusion}

\bibliography{Thesis-Model}

\appendix

\chapter{Demonstration of the Bifurcation Theorem}\label{app:apendiceA}

\end{document}
