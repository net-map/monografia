\documentclass[]{politex}
% REMOVER LINHA ABAIXO PARA VOLTAR À FONTE NORMAL DO LATEX
\renewcommand{\familydefault}{\sfdefault}
% ========== Opções ==========
% pnumromarab - Numeração de páginas usando algarismos romanos na parte pré-textual e arábicos na parte textual
% abnttoc - Forçar paginação no sumário conforme ABNT (inclui "p." na frente das páginas)
% normalnum - Numeração contínua de figuras e tabelas 
%	(caso contrário, a numeração é reiniciada a cada capítulo)
% draftprint - Ajusta as margens para impressão de rascunhos
%	(reduz a margem interna)
% twosideprint - Ajusta as margens para impressão frente e verso
% capsec - Forçar letras maiúsculas no título das seções
% espacosimples - Documento usando espaçamento simples
% espacoduplo - Documento usando espaçamento duplo
%	(o padrão é usar espaçamento 1.5)
% times - Tenta usar a fonte Times New Roman para o corpo do texto
% noindentfirst - Não indenta o primeiro parágrafo dos capítulos/seções


% ========== Packages ==========
\usepackage[utf8]{inputenc}
\usepackage{amsmath,amsthm,amsfonts,amssymb}
\usepackage{graphicx,cite,enumerate}


% ========== Language options ==========
\usepackage[brazil]{babel}
%\usepackage[english]{babel}


% ========== ABNT (requer ABNTeX 2) ==========
%	http://www.ctan.org/tex-archive/macros/latex/contrib/abntex2
%\usepackage[num]{abntex2cite}

% Forçar o abntex2 a usar [ ] nas referências ao invés de ( )
%\citebrackets{[}{]}




% ========== Opções do documento ==========
% Título
\titulo{net.map - Sistema de Posicionamento Indoor}

% Autor
%\autor{Nome Sobrenome}

% Para múltiplos autores (TCC)
\autor{Adriano Dennanni\\
Ricardo Nagano\\
Thiago Lira}

% Orientador / Coorientador
\orientador{Prof. Dr. Reginaldo Arakaki}
\coorientador{Eng. Marcelo Pita}


% Tipo de documento
\tcc{Eletricista com ênfase em Computação}

% Departamento e área de concentração
\departamento{PCS}
\areaConcentracao{Engenharia de Computação}

% Local
\local{São Paulo}

% Ano
\data{2016}




\begin{document}
% ========== Capa e folhas de rosto ==========
\capa
\falsafolhaderosto
\folhaderosto


% ========== Folha de assinaturas (opcional) ==========
%\begin{folhadeaprovacao}
%	\assinatura{Prof.\ X}
%	\assinatura{Prof.\ Y}
%	\assinatura{Prof.\ Z}
%\end{folhadeaprovacao}


% ========== Ficha catalográfica ==========
% Fazer solicitação no site:
%	http://www.poli.usp.br/en/bibliotecas/servicos/catalogacao-na-publicacao.html


% ========== Dedicatória (opcional) ==========
%\dedicatoria{Dedicatória}


% ========== Agradecimentos ==========
\begin{agradecimentos}

\end{agradecimentos}


% ========== Epígrafe (opcional) ==========
\epigrafe{

	\emph{``Anything one man can imagine, other men can make real''}
	\begin{flushright}
		-{}- Jules Verne
	\end{flushright}
	
	\hfill \break
	
	\emph{``Enquanto sentir vontade de competir, buscar desafios e correr atrás de torneios, vou jogar''}
	\begin{flushright}
		-{}- Gustavo Kuerten
	\end{flushright}
	
	\hfill \break
	
	\emph{``Not all those who wander are lost''}
	\begin{flushright}
		-{}- J. R. R. Tolkien
	\end{flushright}
}


% ========== Resumo ==========
\begin{resumo}

Mapas físicos tornam-se cada vez menos utilizados com o desenvolvimento progressivo
de sistemas de posicionamento cada vez melhores. O sistema americano
GPS é possivelmente o mais utilizado, sendo que ele possibilita qualquer um ter informações
sobre sua localização, dando apoio, por exemplo, à praticantes de trilhas e
acampamentos, principalmente em casos de emergência. Porém, em ambientes fechados,
as ondas eletromagnéticas utilizadas pelos satélites sofrem atenuações e interferências
devidos aos materiais de construção, e assim o sistema perde precisão e não
funciona com toda a precisão esperada. Como uma alternativa para esta dificuldade,
procurou-se desenvolver um sistema, que consegue obter a posição do usuário em um
ambiente fechado com precisão, sendo usado para isso técnicas de machine learning,
aliadas com dados obtidos de redes em fio já instaladas no local. O sistema consistirá
de um servidor central, onde serão enviados os dados e os mesmos serão processados.
Os dados serão coletados por meio de um aplicativo de Android, este possuirá
duas versões: versão de usuário final, que usará os dados do servidor para localizá-lo, e a
versão de administrador, que irá coletar dados novos para serem usados em futuras medições.
%
\\[3\baselineskip]
%
\textbf{Palavras-Chave} -- Localização Indoor, Wi-Fi, Machine Learning.
\end{resumo}

% ========== Abstract ==========
\begin{abstract}

Physical maps are becoming each day less used due to constant evolution of positioning
systems, better each day as well. The American system GPS probably is the
most used and the most famous. It allows everyone to have their location information,
giving support to hikers and campers, specially in emergency situations. On the other
hand, in indoor environments, electromagnetic waves used by the satellites suffer with
interference and mitigations and the systems loses precision and does not work as expected.
As an alternative for this difficulty, it was developed a system that can locate
the user position in an indoor environment with precision, using machine learning algorithms
and data of wireless signals collected from the networks already existing on
the place. The system consists on a main server that will receive the data and process
it. The data will be collected with a Android app that will have two versions. The user
version will use the server data to locate the user. The admin version will collect new
data to be user on future measures.
%
\\[3\baselineskip]
%
\textbf{Palavras-Chave} -- Indoor Location, Wi-Fi, Machine Learning.
\end{abstract}

% ========== Listas (opcional) ==========
\listadefiguras
\listadetabelas

% ========== Listas definidas pelo usuário (opcional) ==========
%\begin{pretextualsection}{Lista de símbolos}
%\end{pretextualsection}

% ========== Sumário ==========
\sumario


% ========== Elementos textuais ==========

\chapter{Introdução}\label{chp:introduction}

\section{Motivação}\label{sec:motivation}
Com a modernização das tecnologias de telefonia móvel torna-se cada vez maior o
número de pessoas com acesso à \textit{Internet}, através de tecnologias como \textit{Wi-fi},
3G e 4G. Essas formas de acesso à rede fornecem informações a provedores sobre o
usuário a todo momento,como o conteúdo acessado por seus navegadores ou aplicativos
e informações sobre posição e deslocamento. Dados de localização por si possuem
pouco valor, mas quando aliados a outros conteúdos, é possível fornecer conteúdo
personalizado em tempo real, reativo ao ambiente, passando a oferecer valor real
à empresas e entidades.
\par
Sistemas de posicionamento por satélite como GPS conseguem localizar um dispositivo
na Terra com uma precisão na casa dos centímetros em ambientes abertos. Porém,
o mesmo não ocorre em lugares fechados, como residências e edifícios. Isso ocorre
devido à atenuação dos sinais dos satélites causada pelas paredes e tetos das
estruturas. Tendo em vista o crescimento das cidades e o consequente aumento no
número de construções, as pessoas cada vez passam mais tempo em ambientes
fechados. A necessidade de serviços de localização \textit{indoor} tem se tornado cada
vez mais evidente.
\par
Respondendo a essa necessidade, surgiram alternativas para o posicionamento em
ambientes fechado, tais como o emprego de \textit{tags RFID} (\textit{Radio Frequency Identification}) e do \textit{Bluetooth}. Tendo em vista este
cenário e as condições tecnológicas atuais, este projeto procura apresentar uma
solução alternativa para localização de pessoas em ambientes fechados, como shoppings e
eventos em galpões, sem ter que investir altos valores em infraestrutura. Para tal, será utilizada a tecnologia \textit{Wi-Fi} combinada a técnicas de \textit{Machine Learning}.
\par
A tabela \ref{comparativoLocaliz} compara as tecnologias de localização citadas acima.

\begin{table}[H]
\centering
\caption{Comparativo entre diferentes métodos de localização}
\label{comparativoLocaliz}
\begin{tabular}{c|c|c|c|}
\cline{2-4}
                                                                                                           & \textbf{GPS}                                                               & \textbf{\begin{tabular}[c]{@{}c@{}}Triangulação\\ por Bluetooth\end{tabular}}                                                 & \textbf{\begin{tabular}[c]{@{}c@{}}Wi-Fi e\\ Machine Learning\end{tabular}}                                                                          \\ \hline
\multicolumn{1}{|c|}{\textbf{\begin{tabular}[c]{@{}c@{}}Localização em\\ ambientes fechados \end{tabular}}} & \xmark                                                                     & \cmark                                                                                                                        & \cmark                                                                                                                                               \\ \hline
\multicolumn{1}{|c|}{\textbf{Precisão}}                                                                    & \begin{tabular}[c]{@{}c@{}}Boa\\ (Quando o sinal\\ é estável)\end{tabular} & \begin{tabular}[c]{@{}c@{}}Boa $\sim$ Média\\ (Depende de como\\ foi instalado)\end{tabular}                                  & \begin{tabular}[c]{@{}c@{}}Boa $\sim$ Média\\ (Depende da quantidade\\ de Access Points e\\ medidas no ambiente)\end{tabular}                        \\ \hline
\multicolumn{1}{|c|}{\textbf{Custo}}                                                                       & Uso gratuito                                                               & \begin{tabular}[c]{@{}c@{}}É necessário\\ adquirir, instalar\\ e arcar com a\\ manutenção de\\ Beacons Bluetooth\end{tabular} & \begin{tabular}[c]{@{}c@{}}Custo somente no\\ acesso aos servidores.\\ Leva em consideração\\ que o ambiente já\\ possui Access Points.\end{tabular} \\ \hline
\multicolumn{1}{|c|}{\textbf{\begin{tabular}[c]{@{}c@{}}Gasto de bateria\\ para o usuário\end{tabular}}}   & Alto                                                                       & Médio                                                                                                                         & Baixo                                                                                                                                                \\ \hline
\end{tabular}
\end{table}

Esta abordagem se mostra interessante ao ponto de que sua implementação não
necessita de configurações particulares nas redes ao redor, uma vez que se baseia
em leituras feitas pelo aparelho móvel e no processamento dos dados feitos em um
servidor em nuvem. Tampouco será necessário se conectar a uma desses redes
\textit{Wi-Fi} no ambiente.

\section{Objetivo}\label{sec:objetctive}
O objetivo deste projeto é desenvolver um conjunto de ferramentas que possibilitem
o mapeamento e a identificação de áreas dentro de ambientes fechados. Estas
ferramentas serão utilizadas em dispositivos móveis, possibilitando que os
usuários possam se localizar em locais fechados. Para chegar a esse objetivo, um
sistema de Machine Learning utilizará os valores das potências das redes \textit{Wi-Fi}
presentes nos arredores para aprender a mapear diversas zonas no ambiente.

\section{Justificativa}\label{sec:justify}
Levando em conta a falta de alternativas práticas para sistemas de posicionamento \textit{indoor}, o \textit{net.map} se mostra ideal para suprir essa demanda. O sistema pode ser utilizado por museus (para tornar a experiência mais interativa) ou por \textit{Shopping Centers}  (para sugerir produtos diferentes de acordo com a localização do cliente). Esses dois exemplos de ambientes são tipicamente instalados em ambientes fechados, onde o GPS não funciona bem, e como consequência o \textit{net.map} pode preencher essa lacuna funcionando como o sistema de localização padrão para esses lugares.

\section{Organização}\label{sec:organization}

O restante do documento tem a seguinte estrutura: Na sessão 2 temos uma breve explicação de alguns conceitos fundamentais para o desenvolvimento do trabalho, e uma breve explicação de cada um dos modelos de \textit{Machine Learning} usados. Na sessão 3 é detalhada a especificação do projeto. Na sessão 5 é documentado todo o estudo com o \textit{Machine Learning} e seus respectivos resultados. São apresentados gráficos e justificativas para todo o tratamento e treinamento dos dados.


% ========== Referências ==========
% --- IEEE ---
%	http://www.ctan.org/tex-archive/macros/latex/contrib/IEEEtran
%\bibliographystyle{IEEEbib}

% --- ABNT (requer ABNTeX 2) ---
%	http://www.ctan.org/tex-archive/macros/latex/contrib/abntex2
\bibliographystyle{abntex2-num}

\bibliography{}


% ========== Apêndices (opcional) ==========
\apendice
%\chapter{Alpha}


% ========== Anexos (opcional) ==========
\anexo
%\chapter{}



\end{document}
