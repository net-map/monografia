\documentclass[]{politex}
% REMOVER LINHA ABAIXO PARA VOLTAR À FONTE NORMAL DO LATEX
\renewcommand{\familydefault}{\sfdefault}
% ========== Opções ==========
% pnumromarab - Numeração de páginas usando algarismos romanos na parte pré-textual e arábicos na parte textual
% abnttoc - Forçar paginação no sumário conforme ABNT (inclui "p." na frente das páginas)
% normalnum - Numeração contínua de figuras e tabelas 
%	(caso contrário, a numeração é reiniciada a cada capítulo)
% draftprint - Ajusta as margens para impressão de rascunhos
%	(reduz a margem interna)
% twosideprint - Ajusta as margens para impressão frente e verso
% capsec - Forçar letras maiúsculas no título das seções
% espacosimples - Documento usando espaçamento simples
% espacoduplo - Documento usando espaçamento duplo
%	(o padrão é usar espaçamento 1.5)
% times - Tenta usar a fonte Times New Roman para o corpo do texto
% noindentfirst - Não indenta o primeiro parágrafo dos capítulos/seções


% ========== Packages ==========
\usepackage[utf8]{inputenc}
\usepackage{amsmath,amsthm,amsfonts,amssymb}
\usepackage{graphicx,cite,enumerate}


% ========== Language options ==========
\usepackage[brazil]{babel}
%\usepackage[english]{babel}


% ========== ABNT (requer ABNTeX 2) ==========
%	http://www.ctan.org/tex-archive/macros/latex/contrib/abntex2
%\usepackage[num]{abntex2cite}

% Forçar o abntex2 a usar [ ] nas referências ao invés de ( )
%\citebrackets{[}{]}


% ========== Lorem ipsum ==========
\usepackage{blindtext}



% ========== Opções do documento ==========
% Título
\titulo{net.map - Sistema de Posicionamento Indoor}

% Autor
%\autor{Nome Sobrenome}

% Para múltiplos autores (TCC)
\autor{Adriano Dennanni\\
Ricardo Nagano\\
Thiago Lira}

% Orientador / Coorientador
\orientador{Prof. Dr. Reginaldo Arakaki}
\coorientador{Eng. Marcelo Pita}


% Tipo de documento
\tcc{Eletricista com ênfase em Computação}

% Departamento e área de concentração
\departamento{PCS}
\areaConcentracao{Engenharia de Computação}

% Local
\local{São Paulo}

% Ano
\data{2016}




\begin{document}
% ========== Capa e folhas de rosto ==========
\capa
\falsafolhaderosto
\folhaderosto


% ========== Folha de assinaturas (opcional) ==========
%\begin{folhadeaprovacao}
%	\assinatura{Prof.\ X}
%	\assinatura{Prof.\ Y}
%	\assinatura{Prof.\ Z}
%\end{folhadeaprovacao}


% ========== Ficha catalográfica ==========
% Fazer solicitação no site:
%	http://www.poli.usp.br/en/bibliotecas/servicos/catalogacao-na-publicacao.html


% ========== Dedicatória (opcional) ==========
%\dedicatoria{Dedicatória}


% ========== Agradecimentos ==========
\begin{agradecimentos}

\end{agradecimentos}


% ========== Epígrafe (opcional) ==========
\epigrafe{

	\emph{``Anything one man can imagine, other men can make real''}
	\begin{flushright}
		-{}- Jules Verne
	\end{flushright}
	
	\hfill \break
	
	\emph{``Enquanto sentir vontade de competir, buscar desafios e correr atrás de torneios, vou jogar''}
	\begin{flushright}
		-{}- Gustavo Kuerten
	\end{flushright}
	
	\hfill \break
	
	\emph{``Not all those who wander are lost''}
	\begin{flushright}
		-{}- J. R. R. Tolkien
	\end{flushright}
}


% ========== Resumo ==========
\begin{resumo}

Mapas físicos tornam-se cada vez menos utilizados com o desenvolvimento progressivo
de sistemas de posicionamento cada vez melhores. O sistema americano
GPS é possivelmente o mais utilizado, sendo que ele possibilita qualquer um ter informações
sobre sua localização, dando apoio, por exemplo, à praticantes de trilhas e
acampamentos, principalmente em casos de emergência. Porém, em ambientes fechados,
as ondas eletromagnéticas utilizadas pelos satélites sofrem atenuações e interferências
devidos aos materiais de construção, e assim o sistema perde precisão e não
funciona com toda a precisão esperada. Como uma alternativa para esta dificuldade,
procurou-se desenvolver um sistema, que consegue obter a posição do usuário em um
ambiente fechado com precisão, sendo usado para isso técnicas de machine learning,
aliadas com dados obtidos de redes em fio já instaladas no local. O sistema consistirá
de um servidor central, onde serão enviados os dados e os mesmos serão processados.
Os dados serão coletados por meio de um aplicativo de Android, este possuirá
duas versões: versão de usuário final, que usará os dados do servidor para localizá-lo, e a
versão de administrador, que irá coletar dados novos para serem usados em futuras medições.
%
\\[3\baselineskip]
%
\textbf{Palavras-Chave} -- Localização Indoor, Wi-Fi, Machine Learning.
\end{resumo}

% ========== Abstract ==========
\begin{abstract}

Physical maps are becoming each day less used due to constant evolution of positioning
systems, better each day as well. The American system GPS probably is the
most used and the most famous. It allows everyone to have their location information,
giving support to hikers and campers, specially in emergency situations. On the other
hand, in indoor environments, electromagnetic waves used by the satellites suffer with
interference and mitigations and the systems loses precision and does not work as expected.
As an alternative for this difficulty, it was developed a system that can locate
the user position in an indoor environment with precision, using machine learning algorithms
and data of wireless signals collected from the networks already existing on
the place. The system consists on a main server that will receive the data and process
it. The data will be collected with a Android app that will have two versions. The user
version will use the server data to locate the user. The admin version will collect new
data to be user on future measures.
%
\\[3\baselineskip]
%
\textbf{Palavras-Chave} -- Indoor Location, Wi-Fi, Machine Learning.
\end{abstract}

% ========== Listas (opcional) ==========
\listadefiguras
\listadetabelas

% ========== Listas definidas pelo usuário (opcional) ==========
%\begin{pretextualsection}{Lista de símbolos}
%\end{pretextualsection}

% ========== Sumário ==========
\sumario


% ========== Elementos textuais ==========

\part{Introdução}
	
\chapter{Capítulo com epígrafe}
\capepigrafe[0.5\textwidth]{``Frase espirituosa de um autor famoso''}{Autor famoso}




% ========== Referências ==========
% --- IEEE ---
%	http://www.ctan.org/tex-archive/macros/latex/contrib/IEEEtran
%\bibliographystyle{IEEEbib}

% --- ABNT (requer ABNTeX 2) ---
%	http://www.ctan.org/tex-archive/macros/latex/contrib/abntex2
\bibliographystyle{abntex2-num}

\bibliography{}


% ========== Apêndices (opcional) ==========
\apendice
%\chapter{Alpha}


% ========== Anexos (opcional) ==========
\anexo
%\chapter{}



\end{document}
