\chapter{Considerações Finais}



\section{Não-Escopos e problemas eventuais na criação de um produto}
O sistema apresentado em ultima análise é pouco mais que uma prova de conceito. Não estão incluídos no nosso escopo diversos itens que seriam necessários para que de fato o net.map possa ser comercializado como um produto de fato. Entre elas, falta um desenvolvimento maior nesses seguintes tópicos:


\begin{enumerate}
\item Estudo sobre quais topologias de \textit{Wi-Fi} resultam em uma melhor localização com o nosso sistema
\item Estudo sobre diferentes ganhos de antenas de modelos diversos de celulares e seu impacto na aquisição das potências de sinal 
\item Testes diversos em pelo menos mais 50 prédios para que a eficácia do sistema seja comprovada em qualquer lugar que seja aplicado
\end{enumerate}



\section{Aplicações}

\subsection{\textit{Kit} didático de \textit{IOT} em uma nova disciplina na POLI}
Proposta pelo orientador desse projeto, está sendo estudada a possibilidade da nossa API ser usada em uma nova disciplina na POLI que substituiria a atual disciplina de Engenharia de Software, apresentando conceitos de IOT e afins. A ideia seria que os alunos possam adicionar funcionalidades no sistema net.map de modo a estudar conceitos de arquitetura de sistemas e segurança, por exemplo.

\subsection{Guia para exposições}
O sistema pode ser usado como uma ferramenta de suporte para guias interativos para \textit{smartphones} em exposições. Atualmente, existem softwares de guias que fornecem fotos dos obras, fotos dos museus e até áudios de comentários que podem ser acessados durante a visita. Um exemplo é o museu do Louvre em Paris, que permite que o visitante utilize um \textit{Nintendo 3DS} para ter acesso a uma série de conteúdos extras. Para cada obra, o usuário insere um código utilizando o aparelho e então são disponibilizados áudios, fotos e modelos 3D da obra. O diferencial em usar a solução net.map é que ela possivelmente conseguiria identificar aonde a pessoa se encontra no museu, minimizando a necessidade de inserir muita informação a cerca da exposição, andar e coleção em que a obra se encontra. Isso tornaria a experiência dos visitantes mais fluída e sem \textit{overheads} desnecessários para a aquisição de informações a cerca de uma obra específica.

\section{Comentários finais}

A principal contribuição deste trabalho é fornecer um método alternativo para o problema de posicionamento \textit{indoor}. Ainda há pontos a serem desenvolvidos e uma variedade de testes que devem ser feitos para que a solução desenvolvida possa ser considerada efetivamente completa. Como apontado no capítulo anterior, estes problemas devem ser resolvidos para que possa ser criado um produto a partir do projeto.\par
Mesmo com estas questões é possível assegurar que os pontos positivos da solução desenvolvida serão mantidos, como o baixo custo de implementação e a melhoria de desempenho com o uso continuo do sistema. Por isso, pode-se considerar que de uma visão prática o projeto cumpriu seus objetivos.\par
De um ponto de vista mais amplo, o saldo do projeto foi definitivamente positivo. Foi uma forma de finalizar o curso colocando a prova todos os conceitos aprendidos durante todos os anos de faculdade, desde os fundamentos mais básicos de programação a decisões de arquitetura de software e gerenciamento de projetos.\par
Desta forma, mesmo o sistema desenvolvido apresentando suas limitações, visto a possibilidade ampla de aplicações práticas tanto comerciais como educativas, ele se torna um recurso que se mostra interessante e digno de investimentos futuros.
