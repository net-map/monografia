\chapter{Considerações Finais}



\section{Não-Escopos e problemas eventuais na criação de um produto}
O sistema apresentado em ultima análise é pouco mais que uma prova de conceito. Não estão incluídos no nosso escopo diversos itens que seriam necessários para que de fato o net.map possa ser comercializado como um produto de fato. Entre elas, falta um desenvolvimento maior nesses seguintes tópicos:


\begin{enumerate}
\item Estudo sobre quais topologias de \textit{Wi-Fi} resultam em uma melhor localização com o nosso sistema
\item Estudo sobre diferentes ganhos de antenas de modelos diversos de celulares e seu impacto na aquisição das potências de sinal 
\item Testes diversos em pelo menos mais 50 prédios para que a eficácia do sistema seja comprovada em qualquer lugar que seja aplicado
\end{enumerate}



\section{Uso como \textit{kit} didático de \textit{IOT} em uma nova disciplina na POLI}
Proposta pelo orientador desse projeto, está sendo estudada a possibilidade da nossa API ser usada em uma nova disciplina na POLI que substituiria a atual disciplina de Engenharia de Software, apresentando conceitos de IOT e afins. A ideia seria que os alunos possam adicionar funcionalidades no sistema net.map de modo a estudar conceitos de arquitetura de sistemas e segurança, por exemplo.