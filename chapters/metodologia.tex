\chapter{Metodologia}

\section{Gerenciamento de tarefas}
O projeto desde a sua idealização foi gerenciado com metodologia \textit{Kanbam}, com a equipe sempre desenvolvendo \textit{features} incrementais com pequenos protótipos entregáveis. A ferramenta de gestão utilizada pelo grupo foi o Trello, aplicação de gerenciamento de projetos baseado na web. Foi sempre feita a distinção entre pesquisa, documentação e implementação. Como é típico de metodologias ágeis, os testes sempre foram realizados em paralelo com a implementação de cada nova \textit{feature}, de modo que para cada nova parte do projeto implementada, foram realizados testes para garantir que as funções antigas ainda funcionam individualmente.
\section{Gerenciamento de código}
Para hospedar e versionar os códigos-fonte dos programas foram utilizados repositórios do GitHub, serviço Git baseado em nuvem, gratuito para \textit{softwares Open-Source}. O Git consiste num sistema de controle de versão de software distribuído, com objetivo de minimizar conflitos entre códigos de diferentes contribuidores em um mesmo repositório. Veja mais no anexo \ref{github}.

\section{Pesquisa} 
A pesquisa foi feita simultaneamente com o desenvolvimento, com novas possibilidades sendo estudadas e implementadas, às vezes ao mesmo tempo. Foram de vital importância as ideias obtidas de \textit{papers} nas partes de \textit{ensemble learning} \cite{comparativeEN}, normalização e tratamentos dos dados, bem como a estrutura dos testes de \textit{cross validation} \cite{comparative}, e também na fundamentação teórica do método \textit{Adaboost} \cite{explainingadaboost} e \cite{adaboost}.