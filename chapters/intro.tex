\chapter{Introdução}\label{chp:introduction}

\section{Motivação}\label{sec:motivation}
Com a modernização das tecnologias de telefonia móvel torna-se cada vez maior o
número de pessoas com acesso à \textit{Internet}, através de tecnologias como \textit{Wi-fi},
3G e 4G. Essas formas de acesso à rede fornecem informações a provedores sobre o
usuário a todo momento,como o conteúdo acessado por seus navegadores ou aplicativos
e informações sobre posição e deslocamento. Dados de localização por si possuem
pouco valor, mas quando aliados a outros conteúdos, é possível fornecer conteúdo
personalizado em tempo real, reativo ao ambiente, passando a oferecer valor real
à empresas e entidades.
\par
Sistemas de posicionamento por satélite como GPS conseguem localizar um dispositivo
na Terra com uma precisão na casa dos centímetros em ambientes abertos. Porém,
o mesmo não ocorre em lugares fechados, como residências e edifícios. Isso ocorre
devido à atenuação dos sinais dos satélites causada pelas paredes e tetos das
estruturas. Tendo em vista o crescimento das cidades e o consequente aumento no
número de construções, as pessoas cada vez passam mais tempo em ambientes
fechados. A necessidade de serviços de localização \textit{indoor} tem se tornado cada
vez mais evidente.
\par
Respondendo a essa necessidade, surgiram alternativas para o posicionamento em
ambientes fechado, tais como o emprego de \textit{tags RFID} (\textit{Radio Frequency Identification}) e do \textit{Bluetooth}. Tendo em vista este
cenário e as condições tecnológicas atuais, este projeto procura apresentar uma
solução alternativa para localização de pessoas em ambientes fechados, como shoppings e
eventos em galpões, sem ter que investir altos valores em infraestrutura. Para tal, será utilizada a tecnologia \textit{Wi-Fi} combinada a técnicas de \textit{Machine Learning}.
\par
A seguir podemos ver uma tabela comparando os três métodos de localização citados no parágrafo acima:

\begin{table}[!htb]
\centering
\caption{Comparativo entre diferentes métodos de localização}
\label{comparativoLocaliz}
\begin{tabular}{c|c|c|c|}
\cline{2-4}
                                                                                                           & \textbf{GPS}                                                               & \textbf{\begin{tabular}[c]{@{}c@{}}Triangulação\\ por Bluetooth\end{tabular}}                                                 & \textbf{\begin{tabular}[c]{@{}c@{}}Wi-Fi e\\ Machine Learning\end{tabular}}                                                                          \\ \hline
\multicolumn{1}{|c|}{\textbf{\begin{tabular}[c]{@{}c@{}}Localização em\\ ambientes fechados\end{tabular}}} & \xmark                                                                     & \cmark                                                                                                                        & \cmark                                                                                                                                               \\ \hline
\multicolumn{1}{|c|}{\textbf{Precisão}}                                                                    & \begin{tabular}[c]{@{}c@{}}Boa\\ (Quando o sinal\\ é estável)\end{tabular} & \begin{tabular}[c]{@{}c@{}}Boa $\sim$ Média\\ (Depende de como\\ foi instalado)\end{tabular}                                  & \begin{tabular}[c]{@{}c@{}}Boa $\sim$ Média\\ (Depende da quantidade\\ de Access Points e\\ medidas no ambiente)\end{tabular}                        \\ \hline
\multicolumn{1}{|c|}{\textbf{Custo}}                                                                       & Uso gratuito                                                               & \begin{tabular}[c]{@{}c@{}}É necessário\\ adquirir, instalar\\ e arcar com a\\ manutenção de\\ Beacons Bluetooth\end{tabular} & \begin{tabular}[c]{@{}c@{}}Custo somente no\\ acesso aos servidores.\\ Leva em consideração\\ que o ambiente já\\ possui Access Points.\end{tabular} \\ \hline
\multicolumn{1}{|c|}{\textbf{\begin{tabular}[c]{@{}c@{}}Gasto de bateria\\ para o usuário\end{tabular}}}   & Alto                                                                       & Médio                                                                                                                         & Baixo                                                                                                                                                \\ \hline
\end{tabular}
\end{table}

Esta abordagem se mostra interessante ao ponto de que sua implementação não
necessita de configurações particulares nas redes ao redor, uma vez que se baseia
em leituras feitas pelo aparelho móvel e no processamento dos dados feitos em um
servidor em nuvem. Tampouco será necessário se conectar a uma desses redes
\textit{Wi-Fi} no ambiente.

\section{Objetivo}\label{sec:objetctive}
O objetivo deste projeto é desenvolver um conjunto de ferramentas que possibilitem
o mapeamento e a identificação de áreas dentro de ambientes fechados. Estas
ferramentas serão utilizadas em dispositivos móveis, possibilitando que os
usuários possam se localizar em locais fechados. Para chegar a esse objetivo, um
sistema de Machine Learning utilizará os valores das potências das redes \textit{Wi-Fi}
presentes nos arredores para aprender a mapear diversas zonas no ambiente.

\section{Justificativa}\label{sec:justify}
Levando em conta a falta de alternativas práticas para sistemas de posicionamento \textit{indoor}, o \textit{net.map} se mostra ideal para suprir essa demanda. O sistema pode ser utilizado por museus (para tornar a experiência mais interativa) ou por \textit{Shopping Centers}  (para sugerir produtos diferentes de acordo com a localização do cliente). Esses dois exemplos de ambientes são tipicamente instalados em ambientes fechados, onde o GPS não funciona bem, e como consequência o \textit{net.map} pode preencher essa lacuna funcionando como o sistema de localização padrão para esses lugares.

\section{Organização}\label{sec:organization}

O restante do documento tem a seguinte estrutura: Na sessão 2 temos uma breve explicação de alguns conceitos fundamentais para o desenvolvimento do trabalho, e uma breve explicação de cada um dos modelos de \textit{Machine Learning} usados. Na sessão 3 é detalhada a especificação do projeto. Na sessão 5 é documentado todo o estudo com o \textit{Machine Learning} e seus respectivos resultados. São apresentados gráficos e justificativas para todo o tratamento e treinamento dos dados.