\begin{resumo}

Physical maps are becoming each day less used due to constant evolution of positioning
systems, better each day as well. The American system GPS probably is the
most used and the most famous. It allows everyone to have their location information,
giving support to hikers and campers, specially in emergency situations. On the other
hand, in indoor environments, electromagnetic waves used by the satellites suffer with
interference and mitigations and the systems loses precision and does not work as expected.
As an alternative for this difficulty, it was developed a system that can locate
the user position in an indoor environment with precision, using machine learning algorithms
and data of wireless signals collected from the networks already existing on
the place. The system consists on a main server that will receive the data and process
it. The data will be collected with a Android app that will have two versions. The user
version will use the server data to locate the user. The admin version will collect new
data to be user on future measures.
%
\\[3\baselineskip]
%
\textbf{Palavras-Chave} -- Indoor Location, Wi-Fi, Machine Learning.
\end{resumo}