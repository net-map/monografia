\begin{abstract}

Today we witness the existence of thousands of applications that require the use of user localization to work, such as maps, recommendation applications, augmented reality games, among many others. Most of them use GPS as a localization tool because it is free and accurate in open environments. However, in closed or covered environments, GPS is a bad solution, proving to be inoperative because of the attenuations of the electromagnetic waves of its transmissions.
\par
This project seeks precisely to exploit this weakness of GPS to provide an indoor tracking system for mobile devices by applying Machine Learning algorithms to signal strength data from Wi-Fi networks in the environment around. Thus, it is possible to create various applications that use user's localization indoors, such as shopping malls, event halls and museums.
\par
In this document it will be presented how the learning algorithms work. An application capable of mapping covered installations will also be presented. At last, there is a brief explanation of an augmented reality game that works indoors.
%
\\[3\baselineskip]
%
\textbf{Palavras-Chave} -- Indoor Location, Wi-Fi, Machine Learning.
\end{abstract}