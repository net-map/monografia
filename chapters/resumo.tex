\begin{resumo}

Mapas físicos tornam-se cada vez menos utilizados com o desenvolvimento progressivo
de sistemas de posicionamento cada vez melhores. O sistema americano
GPS é possivelmente o mais utilizado, sendo que ele possibilita qualquer um ter informações
sobre sua localização, dando apoio, por exemplo, à praticantes de trilhas e
acampamentos, principalmente em casos de emergência. Porém, em ambientes fechados,
as ondas eletromagnéticas utilizadas pelos satélites sofrem atenuações e interferências
devidos aos materiais de construção, e assim o sistema perde precisão e não
funciona com toda a precisão esperada. Como uma alternativa para esta dificuldade,
procurou-se desenvolver um sistema, que consegue obter a posição do usuário em um
ambiente fechado com precisão, sendo usado para isso técnicas de machine learning,
aliadas com dados obtidos de redes em fio já instaladas no local. O sistema consistirá
de um servidor central, onde serão enviados os dados e os mesmos serão processados.
Os dados serão coletados por meio de um aplicativo de Android, este possuirá
duas versões: versão de usuário final, que usará os dados do servidor para localizá-lo, e a
versão de administrador, que irá coletar dados novos para serem usados em futuras medições.
%
\\[3\baselineskip]
%
\textbf{Palavras-Chave} -- Localização Indoor, Wi-Fi, Machine Learning.
\end{resumo}