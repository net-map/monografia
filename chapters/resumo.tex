\begin{resumo}

Hoje em dia presenciamos a existência de milhares de aplicações que necessitam o uso da localização do usuário para funcionarem, tais como mapas, aplicativos de recomendação, jogos de realidade aumentada, entre muitos outros. Em sua maioria, utilizam o GPS como ferramenta de localização, por ser gratuito e preciso em ambientes abertos. Entretanto, em ambientes fechados ou cobertos, o GPS se mostra uma péssima solução, se mostrando inoperante devido às atenuações das ondas eletromagnéticas de suas transmissões.
\par
Este projeto busca justamente explorar esse ponto fraco do GPS para oferecer um sistema de localização \textit{indoor} para dispositivos móveis, aplicando algoritmos de \textit{Machine Learning} a dados de intensidade de sinal de redes \textit{Wi-Fi} no ambiente em questão. Assim, é possível criar diversas aplicações que usam a localização do usuário em ambientes fechados, tais como \textit{shopping centers}, galpões de eventos e museus.
\par
Neste documento será apresentado como os algoritmos de aprendizado funcionam. Também será apresentado um aplicativo capaz de mapear instalações cobertas. Por fim, há uma breve explicação de um jogo de realidade aumentada para ambientes fechados.
%
\\[3\baselineskip]
%
\textbf{Palavras-Chave} -- Localização Indoor, Wi-Fi, Machine Learning.
\end{resumo}