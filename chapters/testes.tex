\chapter{Testes \textit{in loco} e Avaliação de Resultados}

\section{Testes \textit{in loco}}

\section{Objetivos e Resultados}
Tendo em vista o principal objetivo proposto para o projeto, que era desenvolver um arcabouço de recursos que possibilitasse o mapeamento e identificação de áreas dentro de ambiente fechados, podemos dizer que o sistema desenvolvido conseguiu alcançá-lo parcialmente.\par
Nas fases iniciais do projeto, as primeiras concepções desenhadas previam a criação de um sistema muito semelhante ao GPS, só que para ambientes fechados. Ou seja, havia a ideia de que o sistema seria capaz de determinar a posição do usuário com a precisão de metros dentro de uma sala. Por exemplo, o sistema seria capaz de determinar se ele está mais perto da janela ou da porta da sala.\par
Porém, como visto anterior anteriormente, durante a implementação do projeto, notou-se que uma limitação que deveria ser considerada: somente é possível determinar se o usuário está ou não presente em determinada área. Desta forma, o sistema não seria capaz de determinar exatamente o ponto em que o usuário se localiza.\par
Assim, com esta limitação, obteve-se como resultado um sistema que, à partir de uma série de leituras de treinamento anteriores dentro de um ambiente, consegue determinar a região em que o usuário se encontra, mas com a condição de que esta região não pode ser somente um ponto dentro do espaço mapeado\par
Além do objetivo principal, durante o desenvolvimento do trabalho surgiu o objetivo de demonstração, para o qual foram desenvolvidos os aplicativos \textit{APScanner} e o \textit{EletricaGO}. Os resultados obtidos a partir destes aplicativos foram usados para demonstrar o funcionamento das ferramentas desenvolvidas e uma das aplicações possível para o futuro.\par
O \textit{APScanner} é importante para ilustrar o processo de aquisição dos sinais e seu envio para o servidor, com o posterior treinamento e a possibilidade imediata de testar o mapeamento feito. Por sua vez, o \textit{EleticaGO} serve para apresentar uma aplicação plausível para o sistema, que usa as ferramentas desenvolvidas por meio de sua API. \par
